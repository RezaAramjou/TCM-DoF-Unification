\documentclass{article}
\usepackage{amsmath}
\usepackage{amssymb}
\usepackage{geometry}
\geometry{a4paper, margin=1in}

\title{Foundational Proofs of the Theory of Characteristic Modes}
\author{Based on Harrington \& Mautz}
\date{}

\begin{document}
\maketitle

\section{Proof of the Symmetry of the Impedance Operator $\mathbf{Z}$}
The symmetry of the operator $\mathbf{Z}$ is a direct consequence of the Lorentz Reciprocity Theorem. 

\paragraph{Thesis} For any two currents $\mathbf{B}$ and $\mathbf{C}$ on a surface $S$, the operator $\mathbf{Z}$ is symmetric, satisfying:
\begin{equation}
    \langle \mathbf{B}, \mathbf{Z}\mathbf{C} \rangle = \langle \mathbf{Z}\mathbf{B}, \mathbf{C} \rangle
\end{equation}
where the symmetric product is defined as $\langle \mathbf{A}, \mathbf{B} \rangle = \int_S \mathbf{A} \cdot \mathbf{B} \,ds$. [cite: 1108]

\paragraph{Proof}
Let a current $\mathbf{B}$ on $S$ produce an electric field $\mathbf{E}_B$, and a current $\mathbf{C}$ on $S$ produce $\mathbf{E}_C$. The reciprocity theorem states:
\begin{equation}
    \int_S \mathbf{C} \cdot \mathbf{E}_B \,ds = \int_S \mathbf{B} \cdot \mathbf{E}_C \,ds
\end{equation}
The electric field generated by a current $\mathbf{J}$ is given by $\mathbf{E} = -L(\mathbf{J})$. [cite: 1105] The operator $\mathbf{Z}$ is the tangential component of $L$. [cite: 1111] Substituting this into the theorem:
\begin{equation}
    \int_S \mathbf{C} \cdot (-\mathbf{Z}\mathbf{B}) \,ds = \int_S \mathbf{B} \cdot (-\mathbf{Z}\mathbf{C}) \,ds
\end{equation}
Using the symmetric product notation, this becomes $\langle \mathbf{C}, -\mathbf{Z}\mathbf{B} \rangle = \langle \mathbf{B}, -\mathbf{Z}\mathbf{C} \rangle$. By linearity, we prove the symmetry:
\begin{equation}
    \langle \mathbf{C}, \mathbf{Z}\mathbf{B} \rangle = \langle \mathbf{B}, \mathbf{Z}\mathbf{C} \rangle \quad \blacksquare
\end{equation}

\section{Proof of Real Eigenvalues ($\lambda_n$) and Eigencurrents ($\mathbf{J}_n$)}
This proof derives from the generalized eigenvalue equation using the real symmetric operators $\mathbf{R}$ and $\mathbf{X}$. 

\paragraph{Thesis} The eigenvalues $\lambda_n$ and eigencurrents $\mathbf{J}_n$ that satisfy the equation $\mathbf{X}(\mathbf{J}_n) = \lambda_n \mathbf{R}(\mathbf{J}_n)$ are purely real. 

\paragraph{Proof of Real Eigenvalues}
Take the complex inner product of the eigenvalue equation with $\mathbf{J}_n$:
\begin{equation}
    \langle \mathbf{J}_n^*, \mathbf{X}\mathbf{J}_n \rangle = \langle \mathbf{J}_n^*, \lambda_n \mathbf{R}\mathbf{J}_n \rangle = \lambda_n \langle \mathbf{J}_n^*, \mathbf{R}\mathbf{J}_n \rangle
\end{equation}
The operators $\mathbf{R}$ and $\mathbf{X}$ are Hermitian, and a property of Hermitian operators is that their quadratic forms, $\langle \psi^*, \mathbf{A}\psi \rangle$, are always real numbers. Since $\langle \mathbf{J}_n^*, \mathbf{X}\mathbf{J}_n \rangle$ and $\langle \mathbf{J}_n^*, \mathbf{R}\mathbf{J}_n \rangle$ are both real, their ratio must be real. Thus, $\lambda_n$ must be real. $\blacksquare$

\paragraph{Proof of Real Eigencurrents}
The eigenvalue equation can be written as $(\mathbf{X} - \lambda_n \mathbf{R})\mathbf{J}_n = 0$. Since $\mathbf{X}$, $\mathbf{R}$, and $\lambda_n$ are all real, the operator $(\mathbf{X} - \lambda_n \mathbf{R})$ is a real symmetric operator. A linear homogeneous equation with a real operator can always possess a set of purely real eigenfunctions $\mathbf{J}_n$.  $\blacksquare$

\section{Proof of Weighted Orthogonality}
\paragraph{Thesis} For two distinct modes $m$ and $n$ ($\lambda_m \neq \lambda_n$), the eigencurrents are orthogonal with respect to both $\mathbf{R}$ and $\mathbf{X}$. 
\begin{align}
    \langle \mathbf{J}_m, \mathbf{R}\mathbf{J}_n \rangle &= 0 \\
    \langle \mathbf{J}_m, \mathbf{X}\mathbf{J}_n \rangle &= 0
\end{align}
\paragraph{Proof}
Consider the eigenvalue equations for modes $m$ and $n$:
\begin{align}
    \mathbf{X}(\mathbf{J}_m) &= \lambda_m \mathbf{R}(\mathbf{J}_m) \label{eq:m} \\
    \mathbf{X}(\mathbf{J}_n) &= \lambda_n \mathbf{R}(\mathbf{J}_n) \label{eq:n}
\end{align}
Take the symmetric product of (\ref{eq:m}) with $\mathbf{J}_n$ and (\ref{eq:n}) with $\mathbf{J}_m$:
\begin{align}
    \langle \mathbf{J}_n, \mathbf{X}\mathbf{J}_m \rangle &= \lambda_m \langle \mathbf{J}_n, \mathbf{R}\mathbf{J}_m \rangle \label{eq:m_prod} \\
    \langle \mathbf{J}_m, \mathbf{X}\mathbf{J}_n \rangle &= \lambda_n \langle \mathbf{J}_m, \mathbf{R}\mathbf{J}_n \rangle \label{eq:n_prod}
\end{align}
Due to the symmetry of $\mathbf{R}$ and $\mathbf{X}$, the left-hand sides of (\ref{eq:m_prod}) and (\ref{eq:n_prod}) are equal. Therefore, the right-hand sides are equal:
\begin{equation}
    \lambda_m \langle \mathbf{J}_m, \mathbf{R}\mathbf{J}_n \rangle = \lambda_n \langle \mathbf{J}_m, \mathbf{R}\mathbf{J}_n \rangle
\end{equation}
Rearranging gives $(\lambda_m - \lambda_n) \langle \mathbf{J}_m, \mathbf{R}\mathbf{J}_n \rangle = 0$. Since $\lambda_m \neq \lambda_n$, it must be that $\langle \mathbf{J}_m, \mathbf{R}\mathbf{J}_n \rangle = 0$. Substituting this back into (\ref{eq:n_prod}) shows that $\langle \mathbf{J}_m, \mathbf{X}\mathbf{J}_n \rangle = 0$. $\blacksquare$

\section{Proof of the Physical Interpretation of $\lambda_n$}
\paragraph{Thesis} $\lambda_n$ is proportional to the difference between the time-average stored magnetic and electric energy for that mode. 
\paragraph{Proof}
The complex Poynting theorem is given by: 
\begin{equation}
    \langle \mathbf{J}^*, \mathbf{Z}\mathbf{J} \rangle = \oint_{S'} \mathbf{E} \times \mathbf{H}^* \cdot d\mathbf{s} + j\omega \iiint_{\tau'} (\mu \mathbf{H}\cdot\mathbf{H}^* - \epsilon \mathbf{E}\cdot\mathbf{E}^*) \,d\tau
\end{equation}
For a normalized eigencurrent $\mathbf{J}_n$, we have $\langle \mathbf{J}_n^*, \mathbf{R}\mathbf{J}_n \rangle = 1$ and $\langle \mathbf{J}_n^*, \mathbf{X}\mathbf{J}_n \rangle = \lambda_n$. Therefore, the left-hand side is $\langle \mathbf{J}_n^*, \mathbf{Z}\mathbf{J}_n \rangle = 1 + j\lambda_n$. 

The real part of the integral term on the right is the radiated power, which is 1 due to normalization. The expression becomes: [cite: 1168, 1169]
\begin{equation}
    1 + j\lambda_n = (1) + j\omega \iiint (\mu |\mathbf{H}_n|^2 - \epsilon |\mathbf{E}_n|^2) \,d\tau
\end{equation}
Equating the imaginary parts of this equation gives the physical meaning of $\lambda_n$:
\begin{equation}
    \lambda_n = \omega \iiint (\mu |\mathbf{H}_n|^2 - \epsilon |\mathbf{E}_n|^2) \,d\tau \quad \blacksquare
\end{equation}
This shows $\lambda_n$ is proportional to the difference between stored magnetic energy ($W_m \propto \int \mu|H|^2 d\tau$) and stored electric energy ($W_e \propto \int \epsilon|E|^2 d\tau$). 

\end{document}