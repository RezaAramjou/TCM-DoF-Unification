\documentclass[11pt]{article}

% PACKAGES
\usepackage[a4paper, margin=1in]{geometry}
\usepackage{amsmath}
\usepackage{amssymb}
\usepackage{amsthm}

% CUSTOM COMMANDS
\newcommand{\vect}[1]{\mathbf{#1}}
\newcommand{\op}[1]{\mathcal{#1}}
\newcommand{\sprod}[2]{\langle #1, #2 \rangle}

% THEOREM ENVIRONMENTS
\newtheorem{theorem}{Theorem}[section]
\theoremstyle{definition}
\newtheorem{definition}{Definition}[section]

% DOCUMENT INFORMATION
\title{\huge \textbf{A Comprehensive Proof of the \\ Theory of Characteristic Modes}}
\author{Derived from the foundational paper by R. F. Harrington and J. R. Mautz \\[1em] \small{IEEE Transactions on Antennas and Propagation, Vol. AP-19, No. 5, September 1971}}
\date{}

\begin{document}

\maketitle

\begin{abstract}
This document provides a detailed, step-by-step derivation of the theory of characteristic modes for conducting bodies. Each concept, equation, and theorem from the foundational paper is proven from first principles to ensure a complete and self-contained explanation. The proofs follow the original operator-based formulation.
\end{abstract}

\hrule\vspace{1.5em}

\section{The Operator Formulation of Electromagnetic Problems}

\subsection{The Integral Equation for Currents}
We begin with the time-harmonic Maxwell's equations, assuming an $e^{j\omega t}$ time dependence:
\begin{align}
\nabla \times \vect{E} &= -j\omega\mu\vect{H} \\
\nabla \times \vect{H} &= j\omega\epsilon\vect{E} + \vect{J}
\end{align}
The electric field $\vect{E}$ can be expressed in terms of the magnetic vector potential $\vect{A}$ and the scalar electric potential $\Phi$:
\begin{equation}
\vect{E} = -j\omega\vect{A} - \nabla\Phi
\end{equation}
For a surface current $\vect{J}$ on a surface $S$, the potentials at a field point $\vect{r}$ are given by integrals over the source points $\vect{r'}$:
\begin{align}
\vect{A}(\vect{r}) &= \mu \oint_S \vect{J}(\vect{r'}) \psi(\vect{r}, \vect{r'}) \,ds' \label{eq:A_potential} \\
\Phi(\vect{r}) &= \frac{-1}{j\omega\epsilon} \oint_S \nabla' \cdot \vect{J}(\vect{r'}) \psi(\vect{r}, \vect{r'}) \,ds' \label{eq:Phi_potential}
\end{align}
where $\psi(\vect{r}, \vect{r'})$ is the free-space Green's function:
\begin{equation}
\psi(\vect{r}, \vect{r'}) = \frac{e^{-jk|\vect{r}-\vect{r'}|}}{4\pi|\vect{r}-\vect{r'}|}
\end{equation}
We define a linear operator, $L$, that maps a surface current $\vect{J}$ to the electric field it produces:
\begin{equation}
L(\vect{J}) = j\omega\vect{A}(\vect{J}) + \nabla\Phi(\vect{J}) \quad [26]
\end{equation}
On a perfect electric conductor (PEC) surface $S$, the tangential component of the total electric field (impressed field $\vect{E}^i$ plus scattered field $\vect{E}^{scat} = L(\vect{J})$) must be zero:
\begin{equation}
[\vect{E}^{i} + L(\vect{J})]_{\text{tan}} = 0
\end{equation}
This gives the fundamental operator equation for the current $\vect{J}$ on $S$:
\begin{equation}
[L(\vect{J})]_{\text{tan}} = -[\vect{E}^{i}]_{\text{tan}} \quad [21]
\end{equation}

\subsection{The Impedance Operator $Z$ and its Properties}
We define an \textbf{impedance operator} $Z$ as the tangential component of the $L$ operator [37]:
\begin{equation}
Z(\vect{J}) = [L(\vect{J})]_{\text{tan}}
\end{equation}
We also define a symmetric product for two vector functions on $S$ [34]:
\begin{equation}
\sprod{\vect{B}}{\vect{C}} = \oint_S \vect{B} \cdot \vect{C} \, ds
\end{equation}

\begin{theorem}[Symmetry of $Z$]
The operator $Z$ is symmetric, i.e., $\sprod{\vect{B}}{Z\vect{C}} = \sprod{Z\vect{B}}{\vect{C}}$ [39].
\end{theorem}
\begin{proof}
The symmetry of $Z$ is a direct consequence of the Lorentz reciprocity theorem. For two currents $\vect{J}_m$ and $\vect{J}_n$ on $S$, the theorem implies $\oint_S \vect{E}_m \cdot \vect{J}_n \, ds = \oint_S \vect{E}_n \cdot \vect{J}_m \, ds$. Since $\vect{E}_{m, \text{tan}} = Z(\vect{J}_m)$ and $\vect{J}_n$ is purely tangential, this becomes:
\[ \oint_S Z(\vect{J}_m) \cdot \vect{J}_n \, ds = \oint_S Z(\vect{J}_n) \cdot \vect{J}_m \, ds \]
In our symmetric product notation, this is $\sprod{\vect{J}_n}{Z\vect{J}_m} = \sprod{\vect{J}_m}{Z\vect{J}_n}$, which proves symmetry.
\end{proof}

We decompose $Z$ into its real and imaginary Hermitian parts, which are real operators because $Z$ is symmetric [45]:
\begin{equation}
Z = R + jX \quad \text{where} \quad R = \frac{1}{2}(Z + Z^*) \quad \text{and} \quad X = \frac{1}{2j}(Z - Z^*) \quad [42, 43]
\end{equation}

\begin{theorem}[Positive Semi-Definiteness of $R$]
The operator $R$ is positive semi-definite [46].
\end{theorem}
\begin{proof}
The time-averaged power radiated by a current $\vect{J}$ is given by $P_{rad} = \frac{1}{2} \text{Re} \{ \sprod{\vect{J}^*}{Z\vect{J}} \}$.
\[ \sprod{\vect{J}^*}{Z\vect{J}} = \sprod{\vect{J}^*}{(R+jX)\vect{J}} = \sprod{\vect{J}^*}{R\vect{J}} + j\sprod{\vect{J}^*}{X\vect{J}} \]
Since $R$ and $X$ are real symmetric operators, $\sprod{\vect{J}^*}{R\vect{J}}$ and $\sprod{\vect{J}^*}{X\vect{J}}$ are real. Thus, $P_{rad} = \frac{1}{2} \sprod{\vect{J}^*}{R\vect{J}}$. Since radiated power is always non-negative, we must have $\sprod{\vect{J}^*}{R\vect{J}} \ge 0$.
\end{proof}

\hrule\vspace{1.5em}

\section{The Characteristic Mode Eigenvalue Problem}

\subsection{Derivation of the Fundamental Eigenvalue Equation}
To find basis functions ($\vect{J}_n$) that diagonalize $Z$, we solve a generalized eigenvalue problem. The key insight is to choose the weighting operator to be $R$, which ensures orthogonality of the radiation patterns [54].
\begin{equation}
Z(\vect{J}_n) = \nu_n R(\vect{J}_n) \quad [51]
\end{equation}
Substituting $Z = R+jX$ and defining the eigenvalue as $\nu_n = 1 + j\lambda_n$ [58]:
\begin{align}
(R+jX)(\vect{J}_n) &= (1+j\lambda_n) R(\vect{J}_n) \quad [57] \\
R(\vect{J}_n) + jX(\vect{J}_n) &= R(\vect{J}_n) + j\lambda_n R(\vect{J}_n) \nonumber
\end{align}
Canceling terms, we arrive at the \textbf{fundamental real-valued eigenvalue equation for characteristic modes}:
\begin{equation}
X(\vect{J}_n) = \lambda_n R(\vect{J}_n) \quad [61]
\end{equation}
Since $R$ and $X$ are real operators, the eigenvalues $\lambda_n$ and the corresponding eigencurrents $\vect{J}_n$ must be real [63].

\subsection{Orthogonality and Normalization of Eigencurrents}

\begin{theorem}[Orthogonality of Eigencurrents]
The eigencurrents $\vect{J}_n$ are orthogonal with respect to the $R$ and $X$ operators for distinct eigenvalues [65, 66].
\end{theorem}
\begin{proof}
Consider two distinct modes, $m$ and $n$ ($\lambda_m \neq \lambda_n$):
\begin{align}
X(\vect{J}_m) &= \lambda_m R(\vect{J}_m) \label{eq:eig_m} \\
X(\vect{J}_n) &= \lambda_n R(\vect{J}_n) \label{eq:eig_n}
\end{align}
Take the symmetric product of (\ref{eq:eig_m}) with $\vect{J}_n$ and (\ref{eq:eig_n}) with $\vect{J}_m$:
\begin{align}
\sprod{\vect{J}_n}{X\vect{J}_m} &= \lambda_m \sprod{\vect{J}_n}{R\vect{J}_m} \\
\sprod{\vect{J}_m}{X\vect{J}_n} &= \lambda_n \sprod{\vect{J}_m}{R\vect{J}_n}
\end{align}
By the symmetry of $R$ and $X$, the left-hand sides are equal, so the right-hand sides must be equal. Using symmetry again on the right-hand side:
\[ (\lambda_m - \lambda_n) \sprod{\vect{J}_m}{R\vect{J}_n} = 0 \]
Since $\lambda_m \neq \lambda_n$, we must have $\sprod{\vect{J}_m}{R\vect{J}_n} = 0$. It immediately follows that $\sprod{\vect{J}_m}{X\vect{J}_n} = 0$ for $m \neq n$.
\end{proof}
The eigencurrents are normalized such that each mode radiates unit power [74]:
\begin{equation}
\sprod{\vect{J}_n^*}{R\vect{J}_n} = 1
\end{equation}
Combining orthogonality and normalization gives the complete orthonormal set of relations [78, 79, 80]:
\begin{align}
\sprod{\vect{J}_m^*}{R\vect{J}_n} &= \delta_{mn} \\
\sprod{\vect{J}_m^*}{X\vect{J}_n} &= \lambda_n \delta_{mn} \\
\sprod{\vect{J}_m^*}{Z\vect{J}_n} &= (1+j\lambda_n) \delta_{mn}
\end{align}

\hrule\vspace{1.5em}

\section{Modal Solutions and Field Properties}

\subsection{Modal Expansion for Current and Fields}
Any current $\vect{J}$ on $S$ can be expanded in the basis of eigencurrents [125]:
\begin{equation}
\vect{J} = \sum_n \alpha_n \vect{J}_n
\end{equation}
Substituting this into $Z(\vect{J}) = \vect{E}^i$ and taking the symmetric product with $\vect{J}_m$:
\[ \sum_n \alpha_n \sprod{\vect{J}_m}{Z\vect{J}_n} = \sprod{\vect{J}_m}{\vect{E}^i} \quad [131] \]
Using the orthogonality relation $\sprod{\vect{J}_m}{Z\vect{J}_n}=(1+j\lambda_n)\delta_{mn}$, the sum collapses:
\[ \alpha_m (1+j\lambda_m) = \sprod{\vect{J}_m}{\vect{E}^i} \equiv V_m^i \quad [133, 135] \]
The term $V_n^i$ is the \textbf{modal excitation coefficient}. The solution for the current is:
\begin{equation}
\vect{J} = \sum_n \frac{V_n^i \vect{J}_n}{1+j\lambda_n} \quad [137]
\end{equation}
The fields are given by corresponding expansions [142, 143]:
\begin{equation}
\vect{E} = \sum_n \frac{V_n^i \vect{E}_n}{1+j\lambda_n} \quad , \quad \vect{H} = \sum_n \frac{V_n^i \vect{H}_n}{1+j\lambda_n}
\end{equation}

\subsection{Physical Interpretation of $\lambda_n$ and Field Orthogonality}

\begin{theorem}[Physical Meaning of $\lambda_n$]
The eigenvalue $\lambda_n$ is $2\omega$ times the net time-averaged stored energy (magnetic minus electric) for mode $n$ [119].
\end{theorem}
\begin{proof}
The complex power for a single normalized mode $\vect{J}_n$ is $\sprod{\vect{J}_n^*}{Z\mathbf{J}_n} = 1 + j\lambda_n$. From the Poynting theorem, the imaginary part of complex power is related to the difference in stored energies:
\[ \text{Im}\{\sprod{\vect{J}_n^*}{Z\mathbf{J}_n}\} = \lambda_n = 2\omega \iiint_{\text{space}} \left( \frac{\mu}{2}|\vect{H}_n|^2 - \frac{\epsilon}{2}|\vect{E}_n|^2 \right) d\tau \quad [24, 116] \]
A positive $\lambda_n$ implies a mode with dominant magnetic energy (inductive), while a negative $\lambda_n$ implies dominant electric energy (capacitive) [417, 418]. A mode with $\lambda_n = 0$ is resonant [419].
\end{proof}

\begin{theorem}[Far-Field Orthogonality]
The characteristic far-fields $\vect{E}_n$ form an orthonormal set on the sphere at infinity, $S_\infty$ [104].
\end{theorem}
\begin{proof}
From the Poynting theorem applied to two modes $m$ and $n$:
\[ \oint_{S_\infty} (\vect{E}_m \times \vect{H}_n^*) \cdot d\vect{s} = \sprod{\vect{J}_m^*}{Z\vect{J}_n} = (1+j\lambda_n)\delta_{mn} \quad [94, 95] \]
In the far-field, $\vect{H}_n = \frac{1}{\eta}(\hat{\vect{r}} \times \vect{E}_n)$. The integral becomes $\frac{1}{\eta} \oint_{S_\infty} (\vect{E}_m \cdot \vect{E}_n^*) ds$. Taking the real part of the equation:
\begin{equation}
\frac{1}{\eta} \oint_{S_\infty} \vect{E}_m \cdot \vect{E}_n^* \, ds = \delta_{mn} \quad [102]
\end{equation}
\end{proof}

\hrule\vspace{1.5em}

\section{Applications and Advanced Formulations}

\subsection{Plane-Wave Scattering}
For an incident plane wave $\vect{E}^i = \vect{u}_i e^{-j\vect{k}_i \cdot \vect{r}}$ [226], the excitation coefficient becomes:
\begin{equation}
V_n^i = \oint_S \vect{J}_n \cdot (\vect{u}_i e^{-j\vect{k}_i \cdot \vect{r}}) \, ds \equiv R_n^i \quad [234]
\end{equation}
The scattered far-field in a direction $(\theta_m, \phi_m)$ with polarization $\vect{u}_m$ is then given by [241]:
\begin{equation}
\vect{E}^s \cdot \vect{u}_m = \frac{-j\omega\mu}{4\pi r_m} e^{-jkr_m} \sum_n \frac{R_n^i R_n^m}{1+j\lambda_n}
\end{equation}
where $R_n^m$ is a similar coefficient for a wave incident from the measurement direction.

\subsection{Diagonalization of the Scattering Matrix}
The scattering operator $S$ relates an incoming wave $E_{in}$ to the resulting outgoing wave $E_{out}$ such that $E_{out} = S E_{in}$ [340]. We choose the characteristic fields $\vect{E}_n$ as the basis for outgoing waves and their conjugates $\vect{E}_n^*$ for incoming waves [345, 348].

\begin{theorem}[Diagonalization of $S$]
In the basis of characteristic fields, the scattering matrix $[S]$ is diagonal [406].
\end{theorem}
\begin{proof}
Consider an impressed field composed of a single mode and its standing wave counterpart, $\vect{E}^i = \vect{E}_m + \vect{E}_m^*$ [386]. Using the convention $\vect{E}_{\text{tan}} = -Z(\vect{J})$, the excitation coefficients are:
\[ V_n^i = \sprod{\vect{J}_n}{-(Z\vect{J}_m + Z^*\vect{J}_m)} = -\sprod{\vect{J}_n^*}{(Z+Z^*)\vect{J}_m} \]
Using orthogonality, this becomes:
\[ V_n^i = - ( (1+j\lambda_m)\delta_{nm} + (1-j\lambda_m)\delta_{nm} ) = -2\delta_{nm} \quad [390] \]
Only the $m$-th coefficient is non-zero, $V_m^i = -2$. The scattered field is therefore:
\[ \vect{E}^s = \sum_n \frac{V_n^i \vect{E}_n}{1+j\lambda_n} = \frac{-2\vect{E}_m}{1+j\lambda_m} \quad [392] \]
The incident wave that produces the impressed field is $E_{in} = \vect{E}_m^*$. The total outgoing wave is $E_{out} = E_{in, \text{transmitted}} + E_{scat} = \vect{E}_m + \vect{E}^s$.
\[ E_{out} = \vect{E}_m + \frac{-2\vect{E}_m}{1+j\lambda_m} = \vect{E}_m \left( 1 - \frac{2}{1+j\lambda_m} \right) = \vect{E}_m \left( \frac{1+j\lambda_m - 2}{1+j\lambda_m} \right) \]
\[ E_{out} = -\frac{1-j\lambda_m}{1+j\lambda_m} \vect{E}_m \]
An incoming wave $\vect{E}_m^*$ produces an outgoing wave proportional only to $\vect{E}_m$. This shows that the scattering matrix $[S]$ is diagonal in this basis, with elements $S_n$:
\begin{equation}
S_n = -\frac{1-j\lambda_n}{1+j\lambda_n} \quad [406]
\end{equation}
This completes the proof, demonstrating that the complex scattering process is decoupled into a series of independent scalar modal responses governed by the real eigenvalues $\lambda_n$.
\end{proof}


\end{document}
